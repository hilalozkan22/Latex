\documentclass[a4paper,10pt]{article}
\usepackage[utf8]{inputenc}
\usepackage{amsmath}
%opening
\title{Rocket Integral}
\author{Hilal Özkan}

\begin{document}

\maketitle

\begin{abstract}

İlk \LaTeX \, çalışmam, ve yükselen bir Roket hesabı içeren bir yazı.

\end{abstract}


 \section{Soru} 
 
 
 Gerçek bir roketin yüksekliği ne olur?

\section{Sorudaki Değişkenler}


\begin{itemize}
 \item $R$ = Roketin boş kütlesi
 \item $Y$ = Yakıtın kütlesi
 \item $T$ = Yakıtı tüketme vakti
 \item $G$ = İtme kuvveti
 \item $g$ = Yerçekimi ivmesi
 \item $t$ = Zaman
\end{itemize}

\section{Roketin Kütlesi}

  İlk olarak t kullanılan bir fonksiyon olarak roketin kütlesi nedir sorusunu yanıtlayacağız.


Roketin kütlesi boş kütle artı yakıttır. t = 0 olunca yakıtın hepsi var demektir , t  = T olunca yakıtın hepsi tüketilmiştir demektir. Lineer bir süreç olduğu için yakıtın tüketilmiş bölüntüsü t/T olur . Dolayısıyla yakıtın kalan bölüntüsü  (1 - t/T)' dir.  Dolayısıyla roketin toplam kütlesi R + Y ( 1 - t/T ) olur.

\[ R + Y ( 1 - \frac{t}{T} )\]


\section{Roketin İvmesi}

Roket üzerindeki kuvvetler : 

\begin{itemize}
\item $1$ . İtme kuvveti
  
\item $2$ . Yerçekimi ivmesi 

\item $3$ .  Roketin ağırlığı

\end{itemize}

Bu kuvvetlerden yola çıkarak roketin ivmesini veren formülü bulabiliriz.

\[
 m.a = G - m.g
\]


Bu formülde bulduğumuz kütle formülünü m yerine koyarsak ve a' yı yalnız bırakırsak ivme formülünü elde ederiz.


\[
a = \frac {G - [ R + Y ( 1 - \frac{t}{T} ) ].g }{R + Y ( 1 - \frac{t}{T})}
\]


\section{Roketin Hızı}

İvme sabit olmadığı için yani değişken olduğu için ivmemizi a= V/t eşitliğinden hesaplayamayız.


Bu yüzden integral kullanarak hesaplayabiliriz :

\[
 V(t) = \int_0^t a(t)dt 
\]

Bu formülde bulduklarımızı yerine koyarsak :

\[
 V(t) = \int_0^t\frac{G - [R + Y (1 - \frac{t}{T})].g}{R + Y (1 - \frac{t}{T} )}dt
\]


Formülünü hesaplarız.

Not :
 \begin{itemize}
  \item $1$ - \[0 <= t <= T\]
  \item $2$ - \[ g > 0\]
  \item $3$ - \[G > 0 \]
  \item $4$ - \[R > 0 \]
  \item $5$ - \[T > 0 \]
  \item $6$ - \[Y > 0\]

  \end{itemize}
    
    olur.

\section{Değişken Değiştirme}

 \[
  t_1 = 1 - \frac{t}{T}
 \]
 
    Değişken değiştirmesini yaparsak;
    
\section{}

Değişken değiştirdikten sonra,

\[
 V(t) = \int_1^{1-\frac{t}{T}}\frac{p-qt_1}{r-st_1} dt_1 = \int_1^{1-\frac{t}{T}}\frac{p}{r-st_1} dt_1 - q \int_1^{1-\frac{t}{T}}\frac{t_1}{r-st_1} dt_1
\]

\[ 
 = p\int_1^{1-\frac{t}{T}}\frac{1}{r-st_1} dt_1 - q \int_1^{1-\frac{t}{T}}\frac{t_1}{r-st_1}dt_1
\]

\[ 
 = p\int_1^{1-\frac{t}{T}}\frac{1}{r-st_1} dt_1 +q \int_1^{1-\frac{t}{T}}\frac{1}{s} dt_1 -\frac{qr}{s}\int_1^{1-\frac{t}{T}}\frac{1}{r-st_1} dt_1
\]

\[
 = \frac{q}{s}\int_1^{1-\frac{t}{T}}{1}dt_1 + \frac{ps-qr}{s}\int_1^{1-\frac{t}{T}}\frac{1}{r-st_1} dt_1
\]

\section{}

\[
\frac{ps-qr}{s}\int_1^{1-\frac{t}{T}}\frac{1}{r-st_1} dt_1
\]

İntegrali için de değişken değiştirmesini yaparsak,
\[
 t_2 = r-st_1
 \]
 \[
 t_1 = \frac{r}{s} - \frac{t_2}{s}
\]
\[
 dt_1 = - \frac{1}{s}dt_2
\]

Değişken değiştirmesinden sonra,

\[
 -(\frac{ps-qr}{s^2})\int_{r-s}^{r-s{(1-\frac{t}{T}})}\frac{1}{t_2}dt_2
\]

\section{}
Bulduğumuz integrali çözelim,

\[
 -(\frac{ps-qr}{s^2})\int_{r-s}^{r-s{(1-\frac{t}{T}})} \frac{1}{t_2}dt_2  = -(\frac{ps-qr}{s^2})[\ln{|{r-s{(1-\frac{t}{T}})}|} - \ln{|r-s|}]
 \]
\[
 = -(\frac{ps-qr}{s^2})[\ln|{\frac{r-s+\frac{st}{T}}{r-s}}|]
\]
\[
 = -(\frac{ps-qr}{s^2})[\ln|1 + \frac{st}{T(r-s)}|]
\]

\section{Roketin Hızı}
  Çözdüğümüz integralde değişkenlerin değerlerini yerine koyarsak roketin hızına ulaşırız. 
  
\[
 V(t) =  -(\frac{ps-qr}{s^2})[\ln|1 + \frac{st}{T(r-s)}|]
\]
\section{Roketin Yüksekliği}
\[ x(t) = \int_{0}^{t}{V(t)}dt
\]

 Formülünde hızın yerine bulduğumuz çözümü yazıp integralini alırsak roketin yüksekliğine ulaşırız.

 \[
 x(t) = \int_{0}^{t}-(\frac{ps-qr}{s^2})[\ln|1 + \frac{st}{T(r-s)}|]dt
 \]
 
 Düzenlersek,
 
 \[
 x(t) = -(\frac{ps-qr}{s^2})\int_{0}^{t}[\ln|1 + \frac{st}{T(r-s)}|]dt
 \]
 
  ln fonksiyonunun integralini almak için kısmi türevden yararlanacağız.
 
 \section{Kısmi Türev}
 
\[
 uv - \int vdu
\]

Formülüne göre değişken değiştirmesi yaparsak,

\[
 u = ln|1 + \frac{st}{T(r-s)}|
\]

\[
 du = \frac{s}{T(r-s) + st}dt
\]

\[
 dv = dt
\]

\[
 v = t
\]

Değişken değiştirdikten sonra,

\[
 tln|1 + \frac{st}{T(r-s)}| - \int_{0}^{t}\frac{st}{T(r-s) + st}dt
\]

Yazdığımız ifadeki integralin çözümü,

\[
   \int_{0}^{t}\frac{st + T(r-s) - T(r-s)}{T(r-s) + st}dt = \int_{0}^{t}( 1 -\frac{T(r-s)}{T(r-s) + st})dt
\]
\[
 = t - \frac{T(r-s)ln|{T(r-s) + st}|}{s} =  \frac{st -T(r-s)ln|{T(r-s) + st}|}{s}
\]

Bulduğumuz cevabı düzenlersek,

\[x(t) =  -(\frac{ps-qr}{s^2})(tln|1 + \frac{st}{T(r-s)}| - \frac{st -T(r-s)ln|{T(r-s) + st}|}{s})
\]

Değişkenlerin değerlerini yerine koyarsak roketin yüksekliğini buluruz.


\end{document}

