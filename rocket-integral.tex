\documentclass[a4paper,10pt]{article}
\usepackage[utf8]{inputenc}
\usepackage{amsmath}
%opening
\title{Rocket Integral}
\author{Hilal Özkan}

\begin{document}

\maketitle

\begin{abstract}

İlk \LaTeX \, çalışmam, ve yükselen bir Roket hesabı içeren bir yazı.

\end{abstract}


 \section{Soru} 
 
 
 Gerçek bir roketin yüksekliği ne olur?

\section{Sorudaki Değişkenler}


\begin{itemize}
 \item $R$ = Roketin boş kütlesi
 \item $Y$ = Yakıtın kütlesi
 \item $T$ = Yakıtı tüketme vakti
 \item $G$ = İtme kuvveti
 \item $g$ = Yerçekimi ivmesi
 \item $t$ = Zaman
\end{itemize}

\section{Roketin Kütlesi}

  İlk olarak t kullanılan bir fonksiyon olarak roketin kütlesi nedir sorusunu yanıtlayacağız.


Roketin kütlesi boş kütle artı yakıttır. t = 0 olunca yakıtın hepsi var demektir , t  = T olunca yakıtın hepsi tüketilmiştir demektir. Lineer bir süreç olduğu için yakıtın tüketilmiş bölüntüsü t/T olur . Dolayısıyla yakıtın kalan bölüntüsü  (1 - t/T)' dir.  Dolayısıyla roketin toplam kütlesi R + Y ( 1 - t/T ) olur.

\[ R + Y ( 1 - \frac{t}{T} )\]


\section{Roketin İvmesi}

Roket üzerindeki kuvvetler : 

\begin{itemize}
\item $1$ . İtme kuvveti
  
\item $2$ . Yerçekimi ivmesi 

\item $3$ .  Roketin ağırlığı

\end{itemize}

Bu kuvvetlerden yola çıkarak roketin ivmesini veren formülü bulabiliriz.

\[
 m.a = G - m.g
\]


Bu formülde bulduğumuz kütle formülünü m yerine koyarsak ve a' yı yalnız bırakırsak ivme formülünü elde ederiz.


\[
a = \frac {G - [ R + Y ( 1 - \frac{t}{T} ) ].g }{R + Y ( 1 - \frac{t}{T})}
\]


\section{Roketin Hızı}

İvme sabit olmadığı için yani değişken olduğu için ivmemizi a= V/t eşitliğinden hesaplayamayız.


Bu yüzden integral kullanarak hesaplayabiliriz :

\[
 V(t) = \int_0^t a(t)dt 
\]

Bu formülde bulduklarımızı yerine koyarsak :

\[
 V(t) = \int_0^t\frac{G - [R + Y (1 - \frac{t}{T})].g}{R + Y (1 - \frac{t}{T} )}dt
\]


Formülünü hesaplarız.

Not :
 \begin{itemize}
  \item $1$ - \[0 <= t <= T\]
  \item $2$ - \[ g > 0\]
  \item $3$ - \[G > 0 \]
  \item $4$ - \[R > 0 \]
  \item $5$ - \[T > 0 \]
  \item $6$ - \[Y > 0\]

  \end{itemize}
    
    olur.

\section{Değişken Değiştirme}

 \[
  t_1 = 1 - \frac{t}{T}
 \]
 
    Değişken değiştirmesini yaparsak;
    
\section{}

Değişken değiştirdikten sonra,

\[
 V(t) = \int_0^1\frac{p-qt_1}{r-st_1} dt_1 = \int_0^1\frac{p}{r-st_1} dt_1 - q \int_0^1\frac{t_1}{r-st_1} dt_1
\]

\[ 
 = p\int_0^1\frac{1}{r-st_1} dt_1 - q \int_0^1\frac{t_1}{r-st_1}dt_1
\]

\[ 
 = p\int_0^1\frac{1}{r-st_1} dt_1 +q \int_0^1\frac{1}{s} dt_1 -\frac{qr}{s}\int_0^1\frac{1}{r-st_1} dt_1
\]

\[
 = \frac{q}{s}\int_0^1{1}dt_1 + \frac{ps-qr}{s}\int_0^1\frac{1}{r-st_1} dt_1
\]


\end{document}
